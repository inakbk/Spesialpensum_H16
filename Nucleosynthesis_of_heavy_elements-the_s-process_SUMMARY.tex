\documentclass[a4paper,12pt]{article}
\usepackage[T1]{fontenc} %for å bruke æøå
\usepackage[utf8]{inputenc}
\usepackage{amsfonts}
\usepackage{amsmath}
\usepackage{amssymb}
%\usepackage[norsk]{babel}
\usepackage{fontenc}
\usepackage{graphicx}
\usepackage{float}
\usepackage{listings}
\usepackage{parskip}
\usepackage[small,bf]{caption}
\usepackage{nuc} %nuclear symbols \Pa{231}
%\usepackage[version=3]{mhchem} for kjemiskte symboler \ce{^{227}_{90}Th+} eller \ce{SO4^2-}

% \usepackage{bm} for bold formler(vektor)


%\def\thesubsection{\arabic{section}.\alph{subsection}}

% to get æ ø å in listings
%\lstset{literate=%
%{æ}{{\ae}}1
%{å}{{\aa}}1
%{ø}{{\o}}1
%{Æ}{{\AE}}1
%{Å}{{\AA}}1
%{Ø}{{\O}}1
%}

\lstset{breaklines}

%fra rapportmal:
\usepackage{verbatim} %for å inkludere filer med tegn LaTeX ikke liker
\usepackage{mathpazo}
\bibliographystyle{plain}

\title{\Huge{\textbf{Summary}} \\ \Large{Nucleosynthesis of heavy elements: the s-process}}
\author{Ina K. B. Kullmann}
\date{\today}
\begin{document}

\maketitle
\part{Nuclear Physics of Stars}
\section*{Chapter 1.7: Nuclear Excited States and Electromagnetic Transitions}
\subsection*{1.7.4: $\gamma$-ray Transistions in a Stellar Plasma}
Excited states are thermally populated. Since the timescale for excitation and de-excitation considerably shorter than the stellar hydrodynamic timescales these excited levels participate in nuclear reactions and $\beta$-decay. 

The number density $N_{\mu}$ of nuclei in excited state $\mu$ divided by the total number density of nuclei $N$ is given by a Boltzmann distrubution:

\begin{equation}
P_\mu = \frac{N_\mu}{N} = \frac{g_\mu e^{-E_\mu/kT}}{\sum_\mu g_\mu e^{-E_\mu/kT}} = \frac{g_\mu e^{-E_\mu/kT}}{G}
\end{equation}
(Assume nondegenerate plasma in thermodynamic equilibrium.)
P is the prob that occupies the excited state $\mu$ for a given isotope. G is partition function is for a given nuclei. 

Thermally excited levels more important (higher prob.) with increasing temperature and lower excitation energy. 

\subsection*{1.7.5: Isomeric States and the Case of \Al{26}}
Half lives ($\gamma$) 
\begin{itemize}
\item Non-isomers: $< 10^{-9}$ s
\item Isomeric/metastable states: isomers: sec, min, days
\end{itemize}

Caused by:
\begin{enumerate}
\item Large difference in spins between the states (large multipolarity, M4, E5)
\item Relatively small energy difference between levels (small $\gamma$-ray energy)
\end{enumerate}
both tend to reduce the decay probability.

\textbf{Ex:} \\
\Al{26} have a isomeric state that would recquire M5 rediation to de-excite to the ground state. More likely the isomeric state can decay by $\beta$-emission to \Mg{26}. Ground state of \Al{26} is also $\beta$-unstable and decays to an excites state of \Mg{26}.

The excited state of \Mg{26} de-excites so quicly that if it is polulated via nuclear reactions in the interiors of stars, the emitted photons would immediately be absorbed by the surrounding matter - would never escape the stellar production site and $\gamma$ never reach earth.

But if \Al{26} is synthesized via nuclear reactions in the stellar interior the long half-life of the ground state gives good opportunity to be expelled from the star into the interstellar medium before decaying to the excited state of \Mg{26}. Then we would be able to see the decay og \Mg{26} on earth. 

We have observed the $\gamma$-lines from \Mg{26} $\Rightarrow$ nucleosynthesis is currently active (since the timescale of \Al{26} half-life is shorter than galactic chemical evolution $\approx 10^{10}$ y.

???
noe noe thermal hmmm

\section*{Chapter 1.8: Weak Interaction}
The strong nuclear force and the electromagnetic force govern the nuclear reactions that are of outstanding importance for the energy generation and the nucleosynthesis in stars. The weak interaction (even w. its short range) also plays a important role for several reasons:
\begin{itemize}
\item decay via weak-interaction processes will compete with its destruction via nuclear reactions.
\item weak interactions determine the \textit{neutron excess parameter} (nr. of excess neutrons per nucleon in the plasma - change only as a result of weak int.)
\item Neutrinos emitted in weak interactions affect the energy budget of stars (influence the models of stellar evolution and explosion)
\end{itemize}

The neutron excess is of crucial importance for nucleosynthesis during the late burning stages in massive stars and during explosive burning:
\begin{equation}
\eta = \sum_i (N_i - Z_i)Y_i = \sum_i \frac{(N_i - Z_i)}{M_i}X_i
\end{equation}
($-1 \leq \eta \leq 1$, $N_i$: nr neutrons, $Z_i$: nr protons, $M_i$: relative atomic mass (in u), $Y_i$: mole fraction, $X_i$ mass fraction (last two def. p. 40)

\subsection*{1.8.1: Weak Interaction Processes}
The weak interaction is very weak. \textbf{Ex:} the free neutron decay is a weak process with $T_{1/2} = 10.2$min which is slower by many orders of magnitude compared to typical nuclear reaction time scales or electromagnetic decay probabilities.  

Weak processes:
\begin{enumerate}
\item $\beta^-$-decay
\item $\beta^+$-decay
\item electron capture
\item neutrino capture
\end{enumerate}
changes chemical identity, but the mass nr A is constant. The leptons do not interact via the strong nuclear force. 4. is used in the detection of solar neutrinos.

\subsection*{1.8.2: Energetics}
The released energy is almost entirely transfered to the emitted leptons. 

For $\beta^\pm$-decay there are three  particles after the interavtion - a three body process - and therefore the neutrino and electron/positron energy distrubutions must be continous, ranging from zero to $Q_\beta$. In electron capture only one lepton is emitted and thus the neutrino is monoenergetic, with $Q_{EC} = E_\nu$ (+ accompanied by  x-ray emission)

$\beta$-delayed particle decays: emission of for instance neutrons in the daghter are delayed by the beta-decay (happening first)

Neutrinos released in $\beta$-decay interact weakly with matter - they are lost from the star unless the density is very large $\rho \geq 10^{11}$ g/cm${}^3 \rightarrow$ the average neutrino energy must usually be subtracted from the total nuclear energy liberated in the energy budget of a star. 

The threshold for pair production may be overcome and the pairs may annihilate again via $e^- + e^+ \rightarrow 2\gamma$ or $e^- + e^+ \rightarrow \nu + \bar{\nu} $ where the neutrinos escape from the star - may be the dominant energy loss at late evolotionary stages (increasing w. temp.).

\subsection*{1.8.3: $\beta$-decay Probabilities}
Focus on elementary Fermi theory of $\beta$-decay: it satisfactory explains lifetimes and shapes of the electron energy dirtributions. The rate of $\beta$-decay can be calculated from the golden rule of time-dependent, first-order pertubation theory. 

\paragraph{Electron or Positron Emission.}
The probability that an electron is emitted depends on the number of final states or the level density and the matrix element $H_{fi}$. The probability is larger if nr. of accecible states is large. The matrix element depends only weakly on energy and determines the overall magnitude of the decay probability and depends on the final and initial wave functions. 

Assume that the emitted neutrino and electron can be treated as free particles (interacts weakly) --> approximate with plane waves and use only the first term in the expansion since the second term usially is very small --> the electron/neutrino wave functions are constant over the nuclear volume. 

A relativistic treatment of $\beta$-decay introduces different matrix elements: \textit{Fermi} and \textit{Gamow-Teller matrix element.} (they depend on the structure of the initial and final states - calculated w. shell model)

Allowed $\beta$-decay: only 1st-order expansion \\
Forbidden decay: degree same as how many terms in the expansion needed before a nonvanishing nuclear matrix element is obtained. (will only look at allowed)

Shape of the electron distrubution is decided by the level density.

Fermi function: the correction for that the electron/positron feels coulomb forces \\
Fermi integral: dimensionless quantity - gives the \textit{ft-value}, experimentally obtained and measure the strength of the particular $\beta$-transition. Google log ft-value??

\paragraph{Electron Capture.}
We use same approach as above, but the electron is not a free particle, so approximate the wave function with the electron wave finction of the K-orbit at the location of the nucleus. The electron capture probability increases strongly with the charce of the parent nucleus. This is why electron capture is greatly favored over positron emission in heavy nuclei. 

\paragraph{Fermi and Gamow-Teller Transitions.}
Forbidden decays: the leptons are required to carry orbital angular momentum.

Allowed decays the leptons do not remove any orbital angular momentum and are subdivided into \textit{Fermi transistions} and \textit{Gamow-Teller transitions} and only occur if certain selection rules are satisfied:
\begin{itemize}
\item Fermi: $\Delta J = |J_i - J_f| = 0$ and $\pi_i = \pi_f$
\item Gamow-Teller:  $\Delta J = |J_i - J_f| = 0$ or $1$ and $\pi_i = \pi_f$ (not $J_i = 0 \rightarrow J_j = 0$)
\end{itemize}
We can have 'pure' transitions and mixed.

In the laboratory $\beta$-decay transitions proceed to all energetically accesible states in the daughter nucleus. Laboratory decay constants and half-lives are also independent of temperature and density. 



\subsection*{1.8.4}


\section*{Chapter 5.6: Nucleosynthesis Beyond the Iron Peak}
s. 514 -  570 samme sidetall i ny og gammel versjon??

(s-prosess viktigst, p- og r- prosess ikke i detalj)

\part{The s-process: Nuclear physics, stellar models, and observations}
38 sider



\part{The Dawes Review 2: Nucleosynthesis and Stellar Yields of Low- and Intermediate-Mass Single Stars}
seksjon 3 og ut (tot er artikkelen 62 sider)





%\begin{figure}[htp]
%\caption{noe}
%\centering
%\includegraphics[width=0.5\textwidth]{tull}
%\label{fig:tull}
%\end{figure}


\end{document}
