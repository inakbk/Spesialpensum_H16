\documentclass[a4paper,12pt]{article}
\usepackage[T1]{fontenc} %for å bruke æøå
\usepackage[utf8]{inputenc}
\usepackage{amsfonts}
\usepackage{amsmath}
\usepackage{amssymb}
%\usepackage[norsk]{babel}
\usepackage{fontenc}
\usepackage{graphicx}
\usepackage{float}
\usepackage{listings}
\usepackage{parskip}
\usepackage[small,bf]{caption}
\usepackage{nuc} %nuclear symbols \Pa{231}
%\usepackage[version=3]{mhchem} for kjemiskte symboler \ce{^{227}_{90}Th+} eller \ce{SO4^2-}

% \usepackage{bm} for bold formler(vektor)


%\def\thesubsection{\arabic{section}.\alph{subsection}}

% to get æ ø å in listings
%\lstset{literate=%
%{æ}{{\ae}}1
%{å}{{\aa}}1
%{ø}{{\o}}1
%{Æ}{{\AE}}1
%{Å}{{\AA}}1
%{Ø}{{\O}}1
%}

\lstset{breaklines}

%fra rapportmal:
\usepackage{verbatim} %for å inkludere filer med tegn LaTeX ikke liker
\usepackage{mathpazo}
\bibliographystyle{plain}

\title{\Huge{\textbf{Summary}} \\ \Large{Nucleosynthesis of heavy elements: the s-process}}
\author{Ina K. B. Kullmann}
\date{\today}
\begin{document}

\maketitle
\part{Nuclear Physics of Stars}
\section*{Chapter 1.7: Nuclear Excited States and Electromagnetic Transitions}
\subsection*{1.7.4: $\gamma$-ray Transistions in a Stellar Plasma}
Excited states are thermally populated. Since the timescale for excitation and de-excitation considerably shorter than the stellar hydrodynamic timescales these excited levels participate in nuclear reactions and $\beta$-decay. 

The number density $N_{\mu}$ of nuclei in excited state $\mu$ divided by the total number density of nuclei $N$ is given by a Boltzmann distrubution:

\begin{equation}
P_\mu = \frac{N_\mu}{N} = \frac{g_\mu e^{-E_\mu/kT}}{\sum_\mu g_\mu e^{-E_\mu/kT}} = \frac{g_\mu e^{-E_\mu/kT}}{G}
\end{equation}
(Assume nondegenerate plasma in thermodynamic equilibrium.)
what is P? the prob that occupies the excited state $\mu$? G is partition function

Thermally excited levels more important (higher prob.) with increasing temperature and lower excitation energy. 

\subsection*{1.7.5: Isomeric States and the Case of \Al{26}}
Half lives ($\gamma$) 
\begin{itemize}
\item Non-isomers: $< 10^{-9}$ s
\item Isomeric/metastable states: isomers: sec, min, days
\end{itemize}

Caused by:
\begin{enumerate}
\item Large difference in spins between the states (large multipolarity, M4, E5)
\item Relatively small energy difference between levels (small $\gamma$-ray energy)
\end{enumerate}
both tend to reduce the decay probability.

\textbf{Ex:} \\
\Al{26} have a isomeric state that would recquire M5 rediation to de-excite to the ground state. More likely the isomeric state can decay by $\beta$-emission to \Mg{26}. Ground state of \Al{26} is also $\beta$-unstable and decays to an excites state of \Mg{26}.

The excited state of \Mg{26} de-excites so quicly that if it is polulated via nuclear reactions in the interiors of stars, the emitted photons would immediately be absorbed by the surrounding matter - would never escape the stellar production site and $\gamma$ never reach earth.

But if \Al{26} is synthesized via nuclear reactions in the stellar interior the long half-life of the ground state gives good opportunity to be expelled from the star into the interstellar medium before decaying to the excited state of \Mg{26}. Then we would be able to see the decay og \Mg{26} on earth. 

We have observed the $\gamma$-lines from \Mg{26} $\Rightarrow$ nucleosynthesis is currently active (since the timescale ogf \Al{26} half-life is shorter than galactic chemical evolution $\approx 10^10$ y.

???
noe noe thermal 

\section*{Chapter 1.8: Weak Interaction}



\subsection*{Chapter 1.8.1}
\subsection*{Chapter 1.8.2}
\subsection*{Chapter 1.8.3}
\subsection*{Chapter 1.8.4}


\section*{Chapter 5.6: Nucleosynthesis Beyond the Iron Peak}
s. 514 -  570 samme sidetall i ny og gammel versjon??

(s-prosess viktigst, p- og r- prosess ikke i detalj)

\part{The s-process: Nuclear physics, stellar models, and observations}
38 sider



\part{The Dawes Review 2: Nucleosynthesis and Stellar Yields of Low- and Intermediate-Mass Single Stars}
seksjon 3 og ut (tot er artikkelen 62 sider)





%\begin{figure}[htp]
%\caption{noe}
%\centering
%\includegraphics[width=0.5\textwidth]{tull}
%\label{fig:tull}
%\end{figure}


\end{document}
